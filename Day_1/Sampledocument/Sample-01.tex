%!TEX TS-program = Arara
% arara: pdflatex: {shell: yes}
\documentclass[12pt,ngerman,parskip=full]{scrreprt}

\usepackage[utf8]{inputenc}
\usepackage[T1]{fontenc}
\usepackage{babel}
\usepackage{blindtext}
\usepackage{graphicx}

%\usepackage[section]{placeins}
%\usepackage[export]{adjustbox}

%\usepackage{fltrace}
%\tracefloatvals
%\usepackage[checktb,addbang]{fewerfloatpages}

\begin{document}

\chapter{Einleitung}
\section{Einführung}
\blindtext[2]

\begin{figure}[h]\centering
\includegraphics[width=0.85\textwidth]{Images/Katze}
\caption{Melli-h}
\end{figure}

\blindtext[2]

\blindtext[2]

\begin{figure}[b]\centering
\includegraphics[width=0.85\textwidth]{Images/Katze}
\caption{Melli-b}
\end{figure}

\blindtext[2]

\blindtext[2]

\chapter{Analyse}

\blindtext[5]

\begin{figure}[t]\centering
\includegraphics[width=0.85\textwidth]{Images/Katze}
\caption{Melli2-t}
\end{figure}

\blindtext[3]

\blindtext[2]

\begin{figure}[p]\centering
\includegraphics[width=0.85\textwidth]{Images/Katze}
\caption{Melli3-p}
\end{figure}

\blindtext[3]

\blindtext[2]

\section{Hellllo}

\begin{figure}[b!]\centering
\includegraphics[width=0.85\textwidth]{Images/Katze}
\caption{Melli4-b!}
\end{figure}

\blindtext[3]

\blindtext[2]

\end{document}