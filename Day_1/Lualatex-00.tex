%!TEX TS-program = Arara
% arara: lualatex: {shell: yes}
\documentclass[12pt,ngerman]{scrartcl}

\usepackage{blindtext}
\usepackage{babel}

\usepackage{fontspec}
\setmainfont{Audiowide}

\begin{document}

\textbf{\blindtext[3]}

\directlua{
a = 5*2.5
tex.print(a)
}

\end{document}

* pdflatex kann nicht mit OpenType Fonts umgehen
* OpenType = Systemschriften unter Win, Mac und Linux
* OpenType => xelatex oder lualatex
* lualatex moderner als xelatex
* Vorteil: OpenType
* Nachteil: Langsamer!
* Vorteil: eingebaute Skriptsprache Lua
