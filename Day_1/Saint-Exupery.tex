%!TEX TS-program = Arara
% arara: lualatex: {shell: yes}
\documentclass[12pt,ngerman,draft]{scrartcl}

\usepackage{fontspec}
\usepackage[german = guillemets]{csquotes}
\usepackage{babel}

\usepackage[left=2cm,right=12.5cm,top=3cm,bottom=3cm]{geometry}
%\usepackage{microtype}
%\usepackage{lua-visual-debug}
%\usepackage{showkerning}

\begin{document}



Als ich im Dezember 1940 durch Portugal reiste, um mich nach den Vereinigten Staaten zu begeben, erschien mir Lissabon als ein lichtes und zugleich trauriges Paradies. Man sprach damals viel von einer drohenden Invasion; Portugal klammerte sich krampfhaft an die Illusion seines Glücks. Lissabon, das die bezauberndste Ausstellung der Welt aufgebaut hatte, lächelte das etwas blasse Lächeln jener Mütter, die von ihrem Sohn im Felde keine Nachricht haben und nun versuchen, ihn durch ihr Vertrauen zu schützen: \enquote{Mein Sohn lebt noch, da ich lächle }

\enquote{Schaut her}, so sagte Lissabon, \enquote{wie glücklich ich bin, wie friedlich, wie gut beleuchtet! ...} Der ganze Kontinent stand drohend um Portugal, gleich einem wilden Gebirge voll räuberischer Stämme; Lissabon, das festliche Lissabon, trotzte: \enquote{Kann man mich zur Zielscheibe machen, da ich mir so viel Mühe gebe, mich nicht zu verstecken! Da ich so verwundbar bin!}

Die Städte meiner Heimat waren des Nachts aschfarben. Ich war bei ihnen auch nicht einen Schimmer mehr gewöhnt, und diese strahlende Hauptstadt bereitete mir ein leises Übelsein. Wenn die Vorstädte finster werden, ziehen die Diamanten einer allzu grell erleuchteten Auslage die Strolche an; man spürt, wie sie sie umkreisen. So fühlte ich um Lissabon die Nacht Europas lasten, von Bombern durchstrichen, als hätten sie den glitzernden Schatz von weitem gewittert.

Aber Portugal ignorierte die Raublust des Ungeheuers; es weigerte sich, an die bösen Zeichen zu glauben. Portugal plauderte über Kunst mit einer verzweiflungsvollen Grimasse von Vertrauensseligkeit. Wird man wagen, es zu zerschmettern, während es sich der Kunst annimmt? Es hatte alle seine Wunder zur Schau gestellt. Würde man wagen, es inmitten seiner Wunder zu zerschmettern? Es zeigte seine großen Männer. In Ermangelung einer Armee, in Ermangelung von Kanonen hatte es gegen das Arsenal des Eroberers alle seine marmornen Schildwachen aufgeführt: die Dichter, die Forscher, die Conquistadoren. Die ganze Vergangenheit Portugals verbarrikadierte die Straße -- in Ermangelung einer Armee und ihrer Kanonen. Würde man wagen, es inmitten des Erbes einer großartigen Vergangenheit zu zerschmettern?

So irrte ich jeden Abend voll Schwermut durch den Triumph dieser Ausstellung äußersten Geschmacks, wo alles an die Vollendung zu streifen schien, bis zu der diskreten Musik, die, mit so viel Takt gewählt, sich sanft in die Gärten ergoß, ohne Lärm, wie das einfache Lied eines Brunnens. Konnte die Welt darauf aus sein, dieses wunderbare Gefühl für das Maß zu zerstören?

Und ich fand Lissabon in seinem Lächeln trauriger als meine erloschenen Städte.

Ich kenne, und vielleicht kennen auch Sie jene etwas sonderbaren Familien, die an ihrem Tisch einem Toten den Platz freihalten. Sie leugnen das Endgültige. Aber nie schien mir dieser Trotz ein Trost zu sein. Tote mußte man dem Tode lassen. Dann wird ihnen, in der Rolle des Totseins, eine andere Form des Daseins zuteil. Jene Familien aber verzögerten ihre Wiederkehr. Sie machten ewig Abwesende aus ihnen, Tischgenossen, die zu spät daran sind für die Ewigkeit. Sie vertauschten die Trauer für ein leeres Warten. Diese Häuser schienen mir in ein hoffnungsloses Unbehagen getaucht, das ganz anders würgt als der Kummer. Um den Flieger Guillaumet, den letzten Freund, den ich verlor und der im Dienste der Flugpost umkam, mein Gott! da hab ich die Trauer auf mich genommen. Guillaumet wird sich nie mehr verändern. Er wird nie mehr da, aber auch nie mehr fort sein. Ich habe sein Gedeck von meinem Tische weggeräumt, diese überflüssige Schlinge, ihn zu fangen, und habe aus ihm einen richtigen toten Freund gemacht.

Aber Portugal zwang sich, an das Glück zu glauben, indem es ihm sein Gedeck ließ, seine Lampions und seine Musik. Man spielte sich in Lissabon ein vermeintliches Glück so intensiv vor, als könne man es Gott selber glaubhaft machen.

Lissabon verdankte seine Atmosphäre der Trauer auch der Anwesenheit gewisser Flüchtlinge. Ich spreche nicht von den Geächteten auf der Suche nach einem Asyl, ich spreche nicht von Emigranten, die nach einem Stück Erde suchen, um es durch ihre Arbeit fruchtbar zu machen. Ich spreche von jenen, die sich aus dem Elend der Ihrigen davonmachten, um ihr Geld zu retten.

Ich hatte in der Stadt kein Quartier gefunden und wohnte in Estoril in der Nähe des Kasinos. Ich kam aus einem unerbittlichen Krieg: meine Staffel hatte durch neun Monate ohne Unterbrechung Deutschland beflogen und im Laufe der deutschen Offensive drei Viertel ihres Bestandes eingebüßt. Zurückgekehrt, hatte ich die unheimliche Atmosphäre der Versklavung und die Drohung des Hungers kennengelernt, hatte die stockfinstere Nacht unserer Städte erlebt. Und siehe! Zwei Schritt von mir füllte sich das Kasino von Estoril jeden Abend mit Gespenstern. Lautlose Cadillacs, die so taten, als führen sie weiß Gott wohin, setzten sie auf dem feinen Sand der Auffahrt ab. Sie hatten sich für das Dinner gekleidet wie ehemals. Sie zeigten ihr Plastron und ihre Perlen. Sie hatten sich gegenseitig zu den Mahlzeiten von Statisten geladen und wußten einander nichts zu sagen ... Dann spielten sie Roulette oder Baccarat, je nach ihrem Vermögen. Ich sah sie mir manchmal an. Ich empfand weder Entrüstung noch das Gefühl von Ironie, wohl aber eine unbestimmte Bangigkeit. Etwa jene, die uns im Zoo vor den Nachzüglern einer ausgestorbenen Gattung überkommt. Sie setzten sich um die Tische. Sie drängten sich um einen kalten, würdig steifen Croupier und mühten sich ab, Hoffnung, Verzweiflung, Angst, Gier und Jubel zu empfinden. Wie Lebende. Sie spielten mit Vermögen, die vielleicht in derselben Minute jede Bedeutung verloren hatten. Sie benützten Münzen, die vielleicht nichts mehr galten. Die Wertpapiere in ihren Kassen waren möglicherweise Papiere von Unternehmungen, die bereits beschlagnahmt waren oder die, von feindlichen Bomben bedroht, eben jetzt zertrümmert und zerstört wurden. Sie zogen Wechsel auf den Sirius. Sich ans Vergangene hängend, zwangen sie sich, an die Rechtmäßigkeit ihres Fiebers, an die Deckung ihrer Schecks, an die Ewigkeit ihrer Formen zu glauben, als hätte es auf dieser Welt nicht zu einer bestimmten Stunde zu krachen angefangen. Es war unwirklich. Es wirkte wie ein Puppenballett. Aber es war traurig. Zweifellos empfanden sie nichts dabei. Ich verließ sie. Ich ging an das Ufer des Meeres, um Atem zu holen. Und dieses Meer von Estoril, das Meer eines Seebades, ein gezähmtes Meer, schien mitzutun bei dem Spiel. Es schob sich in den Golf, eine einzige weiche Woge, ganz silberig vom Mond, die Schleppe eines unzeitgemäßen Kleides.

Ich fand meine Flüchtlinge auf dem Dampfer wieder. Dieses Schiff, ja auch dieses Schiff erzeugte eine leichte Beklemmung. Dieses Schiff brachte lauter Gewächse ohne Wurzeln von einem Kontinent zum andern. Ich sagte mir: \enquote{Ich will gern ein Wanderer sein, aber ich will kein Emigrant sein. Ich habe zu Hause so viele Dinge gelernt, die anderswo unnütz wären.} Da zogen meine Emigranten kleine Notizbücher aus der Tasche; sie bildeten die letzten Reste ihrer Identität. Sie taten noch so, als seien sie wer. Sie hefteten sich mit allen ihren Kräften an irgendeine Bedeutung. \enquote{Sie wissen, ich bin der und der}, sagten sie, \enquote{ich bin aus jener Stadt ..., der Freund eines gewissen ..., kennen Sie einen gewissen ...?} Und sie erzählten einem die Geschichte eines Kumpans, die Geschichte irgendeiner Verantwortlichkeit, die Geschichte einer Verfehlung oder eine andere x-beliebige Geschichte, nur um an irgend etwas Anschluß zu finden. Aber nichts von all dem Vergangenen konnte ihnen helfen, da sie ihr Vaterland verlassen hatten. Es war noch ganz warm, ganz frisch, ganz lebendig, wie es anfangs die Erinnerungen der Liebe sind. Da macht man ein Päckchen aus zärtlichen Briefen. Man fügt ein paar Andenken dazu. Man knüpft alles sorgfältig zusammen. Und anfangs entströmt solchen Reliquien ein melancholischer Zauber. Dann geht eine Blonde mit blauen Augen vorbei, und die Reliquie stirbt. Denn auch der Kumpan, die Verantwortlichkeit, die Geburtsstadt, die Erinnerung an Zuhause verblassen, wenn sie zu nichts mehr nütze sind. Sie fühlten es wohl. So wie Lissabon sich das Glück vorspielte, so spielten sie sich den Glauben vor, bald wieder zurückzukehren. Sie ist ja so süß, die Fremde des verlorenen Sohnes! Es ist eine unechte Fremde, da noch immer das Vaterhaus wartet. Ob man nun ins Nebenzimmer gegangen ist oder auf die andere Seite der Erdkugel: der Unterschied ist unwesentlich. Die Anwesenheit des Freundes, der sich dem Anschein nach entfernt hat, kann fühlbarer werden als seine wirkliche Gegenwart. Es ist jene des Gebetes. Nie habe ich mein Zuhause mehr geliebt als in der Sahara. Nie sind Verlobte ihren Bräuten näher gewesen als die bretonischen Matrosen des XVI. Jahrhunderts, als sie das Kap Horn umsegelten und hinwelkten vor der Mauer undurchdringlicher Winde. Schon vom Augenblick der Abreise an begannen sie heimzukehren. Es war ihre Heimkehr, die sie ins Werk setzten, wenn sie mit schweren Händen die Segel hißten. Der kürzeste Weg vom Hafen in der Bretagne bis zum Hause der Geliebten ging über Kap Horn. Aber meine Emigranten erschienen mir wie bretonische Seefahrer, denen man die bretonische Braut fortgenommen hatte. Keine bretonische Braut zündete für sie im Fenster ihre demütige Lampe an. Sie waren nicht verlorene Söhne. Sie waren verlorene Kinder ohne ein Haus der Heimkehr. Dann erst fängt die wahre Reise an, die Reise aus sich selbst heraus.

Wie sich wiederherstellen? Wie sich die schweren Strähnen der Erinnerungen noch einmal flechten? Dieses Gespensterboot war mit ungeborenen Seelen beladen wie der Vorhimmel. Die einzig Wirklichen – so wirklich, daß man sie gerne mit der Hand berührt hätte – waren diejenigen, die zum Schiffe gehörten und die eine wirkliche Tätigkeit adelte, da sie Tabletts trugen, das Kupfer blank putzten, Stiefel wichsten und mit einer gewissen Herablassung die Leblosen bedienten. Nicht die Armut trug den Emigranten die leise Verachtung des Personals ein. Es fehlte ihnen nicht an Geld, sondern an Gewicht. Es waren nicht mehr Menschen aus einem bestimmten Haus, mit bestimmten Freunden, mit einer bestimmten Verantwortung. Sie spielten die Rolle, aber es war nicht mehr wahr. Niemand brauchte sie, niemand war genötigt, sich an sie zu wenden. Welch ein Wunder ist das Telegramm, das dich durcheinanderrüttelt, dich zwingt, mitten in der Nacht aufzustehn, dich zum Bahnhof jagt: \enquote{Komm schnell! Ich brauche Dich!} Leicht finden wir Freunde, die uns helfen; schwer verdienen wir uns jene, die unsere Hilfe brauchen. Gewiß, niemand haßte meine Emigranten, niemand beneidete sie, niemand belästigte sie. Aber niemand liebte sie mit der einzigen Liebe, die zählt. Ich sagte mir: Sie werden gleich nach ihrer Ankunft zu Willkommcocktails und zu Trostdinners geladen werden. Aber wer wird an ihrer Türe rütteln und Einlaß begehren: \enquote{Öffne! Ich bin's!} Man muß ein Kind lange an der Brust gehabt haben, bis es Forderungen stellt. Man muß sich lange eines Freundes annehmen, ehe er nach der Freundschaft verlangt, die man ihm schuldet. Man muß sich durch Generationen damit zugrunde gerichtet haben, das alte, baufällige Schloß zu retten, um es lieben zu lernen.
II

Ich sagte mir also: \enquote{Das Wesentliche ist, daß das, wovon man gelebt hat, irgendwo weiterbesteht. Und die Gewohnheiten. Und das Familienfest. Und das Haus der Erinnerungen. Das Wesentliche ist, daß man für die Rückkehr lebt.} Und ich fühlte mich bis in den Kern meines Wesens hinein durch die Hinfälligkeit der fernen Pole bedroht, von denen ich abhing; ich riskierte, eine richtige Wüste kennenzulernen, und begann ein Geheimnis zu verstehen, auf das ich schon lange neugierig gewesen war.

Ich habe drei Jahre lang in der Sahara gelebt. Wie so viele andere habe ich über ihren Zauber nachgedacht. Jeder, der das Leben in der Sahara, wo alles nur Armut und Einsamkeit zu sein scheint, kennengelernt hat, weint diesen Jahren als den schönsten des Lebens nach. Die Worte \enquote{Heimweh nach dem Sande, Heimweh nach der Einsamkeit, Heimweh nach dem Raume} sind nur literarische Formeln und erklären nichts. Aber dort an Bord eines Dampfers, der von sich drängenden Passagieren wimmelte, schien es mir zum erstenmal, daß ich die Wüste verstand.

Gewiß, die Sahara ist unabsehbar weit, nur eintöniger Sand – oder genauer, da die Dünen selten sind – ein kieselreicher Strand. Man badet da dauernd im Wesen der Langweile selbst. Indessen bauen ihre unsichtbaren Gottheiten ein Netz von Richtlinien, Neigungen und Zeichen, eine geheimnisvolle und lebendige Muskulatur. Es gibt keine Einförmigkeit mehr. Alles nimmt Richtung an. Keine Stelle gleicht mehr der andern. Es gibt eine Stille des Friedens, wenn die Stämme versöhnt sind, der Abend wieder seine Frische spendet und einem zumute ist, als halte man in einem stillen Hafen mit eingezogenen Segeln Rast. Es gibt eine Stille des Mittags, wenn in der Sonne Gedanken und Bewegungen aussetzen. Es gibt eine falsche Stille, wenn der Nordwind innehält und das Auftauchen von Insekten, die den Oasen des Innern wie Blütenstaub entwehen, den sandführenden Oststurm ankündigt. Es gibt eine Stille der Verschwörung, wenn man von einem entfernten Stamme weiß, daß es in ihm gärt. Es gibt eine geheimnisvolle Stille, wenn sich zwischen den Arabern ihre verschwiegenen Beziehungen anknüpfen. Es ist gespannte Stille, wenn sich die Rückkehr des Boten verzögert.

Eine zugespitzte Stille, wenn man nachts seinen Atem anhält, um zu lauschen. Eine schwermütige Stille, wenn man sich an die erinnert, die man liebt. Alles wird Pol. Jeder Stern bedeutet eine wirkliche Richtung. Es sind alles Sterne der drei Weisen. Sie dienen alle ihrem eigenen Gott. Dieser da bezeichnet die Richtung eines entfernten, schwer erreichbaren Brunnens. Und was dich von diesem Brunnen trennt, ist so gewichtig wie ein Wall. Jener bezeichnet die Richtung eines versiegten Brunnens. Der Stern selbst sieht nach Trockenheit aus. Und was dich von dem versiegten Brunnen trennt, ist kein lockender Hang. Ein anderer Stern dient als Führer zu einer unbekannten Oase, von der dir die Nomaden gesungen haben, die dir aber des Krieges wegen versperrt ist. Und der Sand, der dich von der Oase trennt, ist eine Märchenwiese. Dieser bezeichnet die Richtung einer weißen Stadt im Süden, einer köstlichen, scheint es, köstlich wie eine Frucht, in die man die Zähne schlägt. Und jener die Richtung des Meeres.

Und schließlich wirken von weither die Kräfte fast irrealer Pole wie Magnete in dieser Wüste: ein Haus der Kindheit, das in der Erinnerung lebt. Ein Freund, von dem man nichts weiß, als daß es ihn gibt.

So fühlst du dich gespannt und belebt von dem Feld der Kräfte, die dich anziehen und abstoßen, dich treiben und dir widerstreben. So bist du gut gegründet, gut bestimmt, genau eingesetzt in den Mittelpunkt der Himmelsrichtungen.

Und da die Wüste keinerlei greifbaren Reichtum bietet, da es in ihr nichts zu sehen, nichts zu hören gibt, drängt sich die Erkenntnis auf, daß der Mensch zuvörderst aus unsichtbaren Anreizen lebt, denn das innere Leben, weit entfernt davon einzuschlafen, nimmt an Kräften zu. Der Mensch wird vom Geiste beherrscht. In der Wüste bin ich das wert, was meine Götter wert sind.

So fühlte ich mich an Bord meines traurigen Dampfers reich an noch fruchtbaren Beziehungen; wenn ich einen noch lebenden Planeten bewohnte, so dank meiner Freunde, die ich in Frankreichs Nacht als Verlorene zurückgelassen hatte und die nun begannen, mir wesentlich zu werden.

Frankreich war ohne Frage für mich weder eine abstrakte Göttin noch ein historischer Begriff, sondern ein Leib, zu dem ich gehörte, ein Netz von Bindungen, das mich festhielt, ein Zusammenspiel von Kraftzentren, auf dem die Neigungen meines Herzens beruhten. Ich fühlte das Bedürfnis, diejenigen, deren ich zu meiner Orientierung bedurfte, fester und dauerhafter zu empfinden als mich selbst. Um zu wissen, wohin ich zurückkehre. Um zu leben. In ihnen wohnte mein ganzes Land und lebte durch sie in mir. So stellt sich für einen, der das Meer befährt, ein Kontinent als das stille Blinken einiger Leuchttürme dar. Ein Leuchtturm ist kein Maß für die Entfernung. Sein Licht ist ganz einfach in den Augen gegenwärtig. Und alle Wunder des Kontinents leben in diesem Stern. – Und nun, da Frankreich infolge der totalen Besetzung mit seiner Fracht völlig in das Schweigen eingetreten ist, wie ein Schiff mit gelöschten Feuern, von dem man nicht weiß, hat es die Gefahren des Meeres überlebt oder nicht, nun quält mich das Los all derer, die ich liebe, ärger als eine Krankheit, die sich in mir festgesetzt hätte. Ich sehe mich durch ihre Gefährdung in meinem Wesen bedroht.

Der diese Nacht meine Gedanken heimsucht, ist fünfzig Jahre alt. Er ist krank. Und er ist Jude. Wie wird er den deutschen Terror überstehen? Um mir vorzustellen, daß er noch lebt, bedarf es des Glaubens, daß ihn der Eindringling hinter dem schönen Wall des Schweigens übersehen hat, mit dem ihn die Bauern seines Dorfes schützten. Nur dann glaube ich, daß er noch lebt. Nur dann, wenn ich mich, fern von ihm, im Reich seiner Freundschaft ergehe, das keine Grenzen hat, ist es mir erlaubt, mich nicht als Emigrant zu fühlen, sondern als Wanderer. Denn die Wüste ist nicht da, wo man glaubt. Die Sahara ist lebendiger als eine Hauptstadt, und die volkreichste Stadt wird leer, wenn die wesentlichen Pole des Lebens ihre Kraft einbüßen.
III

Wie baut denn das Leben jene Kraftfelder auf, von denen wir leben? Woher stammt das Gewicht, mit dem es mich zu dem Hause dieses Freundes zieht? Welche sind denn die wesentlichen Momente, die aus dieser Gegenwart einen der Pole gemacht haben, deren ich bedarf? Aus welchen geheimen Vorgängen sind die besonderen Zärtlichkeiten gewoben und aus ihnen wieder die Liebe zur Heimat?

Wie wenig Lärm machen die wirklichen Wunder! Wie einfach sind die wesentlichen Ereignisse! Über den Augenblick, von dem ich erzählen will, gibt es so wenig zu sagen, daß ich ihn träumend wieder erleben und von ihm zu meinem Freunde sprechen muß.

Es war eines Tages vor dem Kriege an den Ufern der Saône, in der Gegend von Tournus. Wir hatten zum Mittagessen ein Restaurant gewählt, dessen Holzveranda über dem Fluß hing. Die Ellbogen auf einen ganz simplen, von Messern zerschnittenen Tisch gestützt, hatten wir uns zwei Pernod bestellt. Dein Arzt hatte Dir den Alkohol verboten, aber zu den großen Gelegenheiten mogeltest Du. Diesmal war es eine. Wir wußten nicht warum, aber es war eine. Was uns da freute, war unwägbarer als die Beschaffenheit des Lichtes. Du hattest Dich also für diesen Pernod der großen Gelegenheiten entschieden. Und, da ein paar Schritte von uns zwei Matrosen einen Kahn löschten, haben wir sie eingeladen. Wir haben sie von unserem Balkon herab angerufen. Und sie sind gekommen. Sie sind ganz einfach gekommen. Wir hatten es so natürlich gefunden, Kumpane einzuladen, vielleicht wegen dieses unsichtbaren Festes in uns. Es war ja so klar, daß sie auf unser Zeichen antworten würden. Und so tranken wir einander zu. Die Sonne tat gut. Die Pappeln des anderen Ufers, die Ebene bis zum Horizont, sie badeten in ihrem linden Honiglicht. Wir wurden immer heiterer und wußten keineswegs, warum. Alles machte uns sicher: die Sonne, die so gut leuchtete, der Fluß, der so schön hinabfloß, das Mahl, das ein richtiges Mahl war; die Matrosen waren auf unseren Zuruf gekommen, das Mädchen bediente uns mit einer Art glücklicher Freundlichkeit, als gäbe sie ein unvergängliches Fest. Wir befanden uns völlig im Frieden, aufs beste eingefügt in eine endgültige Zivilisation, vor Unordnung sicher. Wir genossen eine Art vollkommenen Zustandes, in dem wir uns nichts mehr anzuvertrauen hatten – alle Wünsche waren erfüllt. Wir fühlten uns rein, aufrichtig, klar und milde. Wir hätten nicht zu sagen gewußt, welche Wahrheit es war, deren Evidenz uns entzückte. Aber das Gefühl, das uns beherrschte, war das der Gewißheit. Einer fast übermütigen Gewißheit.

So bewies das Weltall, durch uns hindurch, seinen guten Willen. Die Verdichtung der Nebelflecke, die Erstarrung der Planeten, die Bildung der ersten Amöbe, die ungeheure Arbeit des Lebens auf dem Wege von der Amöbe bis zum Menschen, alles hatte sich glücklich zusammengetan, um, durch uns hindurch, auf diese Stufe der Freude hinauszulaufen! Das war, als Ende der Entwicklung, gar nicht so übel.

So genossen wir dieses stumme Einvernehmen und diese fast religiösen Riten. Gewiegt vom Kommen und Gehen der priesterlichen Magd tranken die Matrosen und wir einander zu wie die Gläubigen ein und derselben Kirche, wenn wir auch nicht hätten sagen können, welcher Kirche. Der eine der beiden Matrosen war Holländer, der andere Deutscher. Dieser war einst dem Nazismus entflohen, weil er da drüben als Kommunist oder als Anhänger Trotzkys verfolgt worden war, oder als Katholik oder als Jude. (Ich erinnere mich nicht mehr der Aufschrift, deretwegen der Mann auf der Liste stand.) Aber in diesem Augenblick war der Matrose etwas ganz anderes als eine Aufschrift. Es ist der Inhalt, der zählt. Die menschliche Substanz. Er war ganz einfach ein Freund. Und wir waren einig unter Freunden. Du warst einig. Ich war einig. Die Matrosen und das Mädchen waren einig. Worüber einig? Über den Pernod? Über die Bedeutung des Lebens? Über die Süße des Tages? Wir hätten auch das nicht zu sagen gewußt. Aber diese Eintracht war so erfüllt, so fest in der Tiefe verankert, sie beruhte auf einer Bibel von so klarem, wenn auch nicht formulierbarem Gehalt, daß wir bereit gewesen wären, das kleine Wirtshaus zu einer Festung zu machen, es zu verteidigen, zu sterben hinter Maschinengewehren, um es zu retten.

Von was für einem Gehalt? ... Gerade hier ist es schwierig, sich auszudrücken! Ich laufe Gefahr, nur den Abglanz einzufangen, nicht das Wesen. Meine Wahrheit wird den unzulänglichen Worten entwischen. Ich wäre undeutlich, wollte ich behaupten, daß wir gern gekämpft hätten, um eine gewisse Art des Lächelns der Matrosen, Deines Lächelns oder meines Lächelns oder des Lächelns der Magd zu retten, ein bestimmtes Wunder dieser Sonne, die sich seit so vielen Millionen Jahren so viel Mühe gegeben hat, um durch uns in ein völlig geglücktes Lächeln zu münden.

Das Wesentliche hat meistens kein Gewicht. Hier war das Wesentliche, allem Anschein nach, nur ein Lächeln. Ein Lächeln ist oft das Wesentliche. Man wird mit einem Lächeln bezahlt. Man wird mit einem Lächeln belohnt. Man wird durch ein Lächeln belebt. Und die Art eines Lächelns kann Schuld daran sein, daß man stirbt. Uns hat seine Art indessen so gut von der Angst unserer Zeit erlöst, uns Sicherheit, Hoffnung und Frieden gewährt, daß ich, um verständlicher zu werden, noch die Geschichte eines anderen Lächelns erzählen muß.
IV

Es war im Verlauf einer Reportage über den Bürgerkrieg in Spanien. Ich hatte die Unklugheit begangen, mich in einen Güterbahnhof einzuschmuggeln, um morgens um 3 Uhr dem Verladen geheimen Kriegsmaterials beiwohnen zu können. Die Bewegung der Mannschaften und eine gewisse Dunkelheit begünstigten mein Vorhaben. Aber ich schien den anarchistischen Soldaten verdächtig zu sein.

Es war sehr einfach. Ich ahnte noch nichts von ihrem geschmeidigen und geräuschlosen Näherkommen, als sie mich schon umschlossen, sanft wie die Finger einer Hand. Der Lauf eines Karabiners richtete sich leicht gegen meinen Bauch, und die Stille schien mir feierlich. Ich hob schließlich die Arme.

Ich beobachtete, daß sie nicht in mein Gesicht, sondern auf meine Krawatte starrten (die Mode einer anarchistischen Vorstadt ließ diesen Kunstgegenstand nicht geraten erscheinen). Meine Haut überflog es. Ich erwartete den Schuß, es war die Zeit der flinken Urteile. Aber es kam kein Schuß. Nach Sekunden einer absoluten Leere, in deren Verlauf es mir schien, als tanzte die arbeitende Mannschaft auf einem andern Stern eine Art Traumballett, gaben mir meine Anarchisten mit einer leichten Kopfbewegung das Zeichen, ihnen voranzugehen, und wir setzten uns ohne Hast über die Verschubgleise in Marsch. Die Gefangennahme hatte sich in vollkommenem Schweigen abgespielt und mit außerordentlicher Sparsamkeit in der Bewegung. So spielt die Tierwelt der Tiefsee.

Bald verschwand ich in einem Kellerloch, aus dem man eine Wachtstube gemacht hatte. Elend beleuchtet von einer schlechten Petroleumlampe, dösten dort andere Milizsoldaten, ihre Karabiner zwischen den Beinen. Sie wechselten mit unbeteiligter Stimme ein paar Worte mit den Männern meiner Patrouille. Einer von ihnen durchsuchte mich.

Ich spreche spanisch, aber ich kann nicht katalanisch. Ich verstand jedoch, daß man meine Papiere verlangte. Ich hatte sie im Hotel vergessen. Ich antwortete: \enquote{Hotel ... Journalist ...}, ohne zu erkennen, ob diese meine Sprache als Mittel der Verständigung taugte.

Die Milizsoldaten reichten – wie ein Beweisstück – meinen Photoapparat von Hand zu Hand. Einige von den Gähnenden, die auf ihren krummbeinigen Sesseln zusammengesunken waren, richteten sich in einer Art Langweile auf und lehnten sich an die Mauer. Der vorherrschende Eindruck war der der Langweile. Der Langweile und des Schlafes. Das Aufmerksamkeitsvermögen dieser Männer schien mir längst verbraucht. Fast hätte ich mir ein Zeichen der Feindseligkeit gewünscht, nur um menschlichen Kontakt zu spüren. Aber sie würdigten mich weder eines Zeichens von Zorn noch eines der Mißbilligung. Ich versuchte zu wiederholten Malen, auf Spanisch zu protestieren. Meine Proteste trafen ins Leere. Sie sahen mich an, ohne darauf einzugehen, so wie sie einen chinesischen Fisch in einem Aquarium angeschaut hätten. Sie warteten. Worauf warteten sie? Auf die Rückkehr eines Genossen? Auf das Morgengrauen? Ich sagte mir: \enquote{Sie warten vielleicht darauf, Hunger zu haben ...}

Ich sagte mir auch: \enquote{Sie werden eine Dummheit machen! Es ist einfach lächerlich ...} Das Gefühl, das ich empfand, war – viel mehr als ein Gefühl der Angst – der Ekel vor dem Abgeschmackten. Ich sagte mir: \enquote{Wenn sie auftauen, wenn sie handeln wollen, werden sie schießen!}

War ich wirklich in Gefahr, ja oder nein? Nahmen sie noch immer nicht zur Kenntnis, daß ich weder ein Saboteur noch ein Spion war, sondern ein Journalist? Daß sich meine Ausweispapiere im Hotel befanden? Hatten sie sich entschieden? Wofür?

Ich wußte nichts von ihnen, außer daß sie ohne große Gewissenskämpfe füsilierten. Die revolutionären Stoßtrupps, gleichgültig, welcher Partei sie angehören, machen nicht Jagd auf Menschen (sie wägen den Menschen nicht nach seiner Substanz), sondern auf Symptome. Die gegnerische Wahrheit erscheint ihnen als epidemische Krankheit. Um eines zweifelhaften Anzeichens willen schickt man den Bazillenträger in das Isolierungslager. Den Friedhof. Darum schien mir dieses Verhör unheilvoll, das mich von Zeit zu Zeit in undeutlicher Einsilbigkeit traf und von dem ich nichts verstand. Ein blindes Roulette spielte um meine Haut. Deshalb empfand ich auch das wunderliche Bedürfnis, ihnen über mich etwas zuzurufen, das mich in mein eigentliches Schicksal hineintreiben würde – nur um das Gewicht einer wirklichen Gegenwart zu spüren. Mein Alter zum Beispiel! Doch, das Alter eines Menschen ist eindrucksvoll. Es enthält sein ganzes Leben. Die Reife, die nun sein ist, ist langsam entstanden. Sie hat sich gegen so viele nun überwundene Hindernisse gebildet, gegen so viele schwere, nun wieder geheilte Krankheiten, gegen so viele gestillte Schmerzen, überwundene Verzweiflungen, gegen Gefahren, von denen die meisten dem Bewußtsein entgangen sind. Sie ist quer durch Wünsche, Hoffnungen und Sehnsüchte, durch viel Vergessen und viel Liebe hindurch gewachsen. Ja, das Alter eines Menschen, es bedeutet eine schöne Fracht von Erfahrungen und Erinnerungen! Trotz der Fallen, der Stöße, der Räderspuren hat man wohl oder übel seinen Weg verfolgt, holterdipolter wie ein guter Karren. Und jetzt, dank eines eigensinnigen Zusammenspiels glücklicher Umstände, ist man soweit. Man ist siebenunddreißig Jahre alt. Und der gute Karren wird, so Gott will, seine Last von Erinnerungen noch weiter schleppen. Ich sagte mir also: \enquote{Soweit bin ich nun. Ich bin siebenunddreißig Jahre alt ...} Ich hätte meine Richter gerne mit dieser vertraulichen Mitteilung belästigt ..., aber sie verhörten mich nicht mehr.

Da war es, daß sich das Wunder begab. Oh, ein sehr verschwiegenes Wunder. Ich hatte keine Zigaretten mit. Da einer meiner Kerkermeister rauchte, bat ich ihn mit einer kleinen Bewegung, mir eine abzutreten, und ich versuchte ein vages Lächeln. Der Mann reckte sich zuerst, führte langsam die Hand an seine Stirn, hob die Augen, so daß er nicht mehr auf meine Krawatte, sondern in mein Gesicht blickte, und zu meiner größten Verblüffung machte auch er den Versuch eines Lächelns. Es war wie der Anbruch des Tages.

Dieses Wunder löste das Drama nicht, sondern schaffte es ganz einfach aus der Welt – wie das Licht den Schatten. Es gab kein Drama mehr. Dieses Wunder änderte nichts, was man hätte sehen können. Die schlechte Petroleumlampe, der Tisch mit verstreuten Papieren, die an die Mauer gelehnten Männer, die Farbe der Gegenstände, der Geruch: alles blieb so, wie es war. Aber jedes Ding war bis in seinen Kern verwandelt. Dieses Lächeln machte mich frei. Es war ein ebenso endgültiges, in seinen Folgen selbstverständliches und nicht mehr umkehrbares Ereignis wie die Erscheinung der Sonne. Es öffnete den Zutritt zu etwas Neuem. Nichts hatte sich geändert, alles war verwandelt. Der Tisch mit den zerstreuten Papieren lebte, die Petroleumlampe lebte, die Mauern lebten. Die Langweile, die aus den toten Gegenständen dieses Kellerloches sickerte, verflüchtigte sich wie durch Zauberei. Es war, als hätte ein unsichtbares Blut wieder zu kreisen begonnen, das alle Dinge zu einem einzigen Körper zusammenband und ihnen so ihre Bedeutung wieder zurückgab.

Auch die Männer hatten sich nicht gerührt, aber während sie mir noch vor einem Augenblick entfernter erschienen waren als vorsintflutliche Geschöpfe, rückten sie nun in lebendige Nähe. Ich hatte einen außergewöhnlichen Eindruck von Gegenwart. So ist es: von Gegenwart! Und ich fühlte mich verwandt.

Der Junge, der gelächelt hatte und der eine Sekunde vorher nur eine Funktion, ein Werkzeug, eine Art riesiges Insekt gewesen war, ließ sich ein bißchen linkisch an, beinahe schüchtern, von einer wunderbaren Schüchternheit. Nicht daß er weniger brutal als ein anderer gewesen wäre, dieser Terrorist! Aber die Geburt des Menschen in ihm machte sein verwundbares Teil so hell. Wir geben uns ein großartiges Ansehen, wir Menschen, aber heimlich im Herzen kennen wir das Zögern, den Zweifel, den Kummer ...

Noch war nichts gesagt worden. Aber alles war entschieden. Ich legte meine Hand dankend auf die Schulter des Milizsoldaten, als er mir die Zigarette reichte. Das Eis war gebrochen; und da nun auch die anderen Soldaten wieder Menschen geworden waren, trat ich in das Lächeln aller ein, wie in ein neues und freies Land.

Ich trat in ihr Lächeln ein wie ehemals in das Lächeln unserer Retter aus der Sahara. Als uns die Kameraden nach tagelangem Suchen gefunden hatten und so nahe als möglich gelandet waren, gingen sie mit großen Schritten auf uns zu, wobei sie die Wasserschläuche mit ausgestreckten Armen gut sichtbar schwenkten. An das Lächeln der Retter, wenn ich schiffbrüchig, an das Lächeln der Schiffbrüchigen, wenn ich Retter war, denke ich wie an meine Heimat, in der ich mich glücklich fühlte. Die wahre Freude ist die Freude am andern. Die Rettung war nichts als eine Gelegenheit zu dieser Freude. Das Wasser hat erst dann die Kraft zu beglücken, wenn es zuvor das Geschenk des guten Willens eines Menschen ist.

Die Sorge für einen Kranken, die Aufnahme eines Geächteten, selbst die Verzeihung haben ihren Wert nur von Gnaden des Lächelns, das die Feier erhöht. Wir vereinigen uns im Lächeln über allen Sprachen, Kasten, Parteien. Wir sind die Gläubigen ein- und derselben Kirche, er mit seinen Bräuchen, und ich mit den meinen.
V

Ist diese Art der Freude nicht die kostbarste Frucht unserer Gesittung? Auch eine totale Tyrannei könnte uns in unseren materiellen Bedürfnissen befriedigen. Aber wir sind nicht Vieh zum Mästen. Unser Gedeihen, unser Behagen, sie würden nicht genügen, uns glücklich zu machen. Für uns, die wir im Kult der Ehrfurcht vor dem Menschen aufgewachsen sind, wiegen die einfachen Begegnungen schwer, die sich manchmal in wunderbare Feste verwandeln ...

Ehrfurcht vor dem Menschen! Ehrfurcht vor dem Menschen! Wenn der Nazist ausschließlich den respektiert, der ihm gleicht, dann respektiert er nur sich selbst. Er verneint die schöpferischen Gegensätze, zerstört jede Hoffnung auf einen Aufstieg und begründet für tausend Jahre an Stelle des Menschen den Robot eines Termitenhaufens. Ordnung um der Ordnung willen beschneidet den Menschen seiner wesentlichen Kraft, der nämlich, die Welt und sich selber umzuformen. Das Leben schafft Ordnung, aber die Ordnung bringt kein Leben hervor.

Es scheint uns im Gegenteil, daß unser Aufstieg noch nicht vollendet ist, daß die morgige Wahrheit sich vom gestrigen Irrtum nährt, und daß die zu überwindenden Gegensätze für unser Wachstum der rechte Humus sind. Wir zählen auch die zu den unsrigen, die anders sind als wir. Aber welch merkwürdige Verwandtschaft! Sie gründet sich auf das Künftige, nicht auf das Vergangene. Auf das Endziel, nicht auf den Ausgangspunkt. Wir sind einer für den andern Pilger, die auf verschiedenen Wegen einem gemeinsamen Treffpunkt zuwandern.

Aber heute ist der Respekt vor dem Menschen, diese Voraussetzung unserer Entwicklung, in Gefahr. Der Zerfall der modernen Welt hat uns ins Finstre geschleudert. Die Probleme hängen nicht mehr zusammen, die Lösungen widersprechen sich. Die Wahrheit von gestern ist tot, die von morgen erst zu gebären. Noch ist keine gültige Synthese vorauszusehen, und jeder von uns hält nur ein Teilchen der Wahrheit in Händen. In Ermangelung zwingender Evidenz nehmen die politischen Religionen ihre Zuflucht zur Gewalt. Und während wir uns so über die Methoden streiten, laufen wir Gefahr, nicht mehr zu erkennen, daß wir auf dem Weg zum gleichen Ziele sind.

Der Wanderer, der seinen Berg in der Richtung eines Sternes überschreitet, läuft Gefahr zu vergessen, welcher Stern ihn führt, wenn er sich zu sehr von den Fragen des Anstieges gefangen nehmen läßt. Wenn er nur noch handelt, um zu handeln, wird er nirgends hinkommen. Die Kirchenstuhlvermieterin einer Kathedrale, die sich zu eifrig mit dem Vermieten der Kirchenstühle befaßt, läuft Gefahr zu vergessen, daß sie einem Gotte dient. Wenn ich mich an irgendeine Parteileidenschaft verliere, laufe ich Gefahr zu vergessen, daß die Politik nur dann einen Sinn hat, wenn sie im Dienst einer geistigen Gewißheit steht. Wir haben in den Stunden des Wunders eine ganz bestimmte Beschaffenheit der menschlichen Beziehungen verkostet: da liegt für uns die Wahrheit.

Wie dringlich eine Handlung auch sein mag, wir dürfen nie vergessen, daß eine innere Berufenheit sie beherrschen muß, soll sie nicht unfruchtbar bleiben. Wir wollen die Ehrfurcht vor dem Menschen begründen. Warum sollen wir uns innerhalb ein- und desselben Lagers hassen? Keiner von uns besitzt das Monopol auf die Reinheit der Absichten. Ich kann im Namen meines Weges den Weg bekämpfen, den ein anderer gewählt hat. Ich kann die Schritte seines Verstandes kritisieren, das Verfahren des Verstandes ist unsicher. Aber ich muß auf der Ebene des Geistes den Mann achten, der nach dem gleichen Stern strebt.

Ehrfurcht vor dem Menschen! Ehrfurcht vor dem Menschen! ... Wenn die Ehrfurcht vor dem Menschen in den Herzen der Menschen wurzelt, werden die Menschen einmal so weit kommen, ihrerseits wieder das soziale, politische oder ökonomische System zu begründen, das diese Ehrfurcht für immer gewährleistet. Eine Zivilisation bildet sich zuerst im Kern. Sie ist im Menschen zuerst das blinde Verlangen nach einer gewissen Wärme. Von Irrtum zu Irrtum findet der Mensch den Weg zum Feuer.
VI

Darum, mein Freund, brauche ich so sehr Deine Freundschaft. Ich dürste nach einem Gefährten, der, jenseits der Streitfragen des Verstandes, in mir den Pilger dieses Feuers sieht. Ich habe das Bedürfnis, manchmal die künftige Wärme vorauszukosten und mich auszuruhen, ein bißchen außerhalb meiner selbst, in der Zusammenkunft, die wir haben werden. Ich bin aller Streite, aller Abschließungen, aller Glaubenswut so müde! Zu dir kann ich kommen, ohne eine Uniform anziehen oder einen Koran hersagen zu müssen; kein Stück meiner inneren Heimat brauche ich preiszugeben. In Deiner Nähe habe ich mich nicht zu entschuldigen, nicht zu verteidigen, brauche ich nichts zu beweisen; ich finde den Frieden wie in Tournus. Über meine ungeschickten Worte, über die Urteile hinweg, die mich irreführen können, siehst Du in mir einfach den Menschen. Du ehrst in mir den Boten eines Glaubens, gewisser Gewohnheiten und besonderer Zuneigungen. Wenn ich auch anders bin als Du, so bin ich doch weit davon entfernt, Dich zu beeinträchtigen; ich steigere Dich vielmehr. Du befragst mich, wie man den Reisenden befragt.

Ich, der ich wie jeder das Bedürfnis empfinde, erkannt zu werden, ich fühle mich in Dir rein und gehe zu Dir. Ich muß dorthin gehen, wo ich rein bin. Weder meine Bekenntnisse noch meine Haltung haben Dich darüber belehrt, wer ich bin. Dein Jasagen zu dem, was ich bin, hat Dich gegen Haltung und Bekenntnis nachsichtig gemacht, so oft es nötig war. Ich weiß Dir Dank dafür, daß Du mich so hinnimmst, wie ich bin. Was habe ich mit einem Freund zu tun, der mich wertet? Wenn ich einen Hinkenden zu Tisch lade, bitte ich ihn, sich zu setzen, und verlange von ihm nicht, daß er tanze.

Mein Freund, ich brauche Dich wie eine Höhe, in der man anders atmet! Ich möchte mich noch einmal neben Dir mit den Ellbogen auf den Tisch stützen, an den Ufern der Saône, auf den Tisch einer kleinen, wackeligen Bretterbude, und zwei Matrosen einladen, in deren Gesellschaft wir einander zutrinken würden, im Frieden eines Lächelns, das wie der Tag ist.

Wenn ich noch kämpfe, werde ich ein wenig auch für Dich kämpfen. Ich brauche Dich, um an die Wiederkunft dieses Lächelns besser glauben zu können. Ich muß Dir helfen dürfen zu leben. Ich sehe Dich so schwach, so bedroht, sehe Dich vor dem Eingang irgendeines Kramladens Deine fünfzig Jahre stundenlang mitschleppen, in dem fadenscheinigen Mantel schlotternd vor Kälte. Ich fühle Dich, der Du so sehr Franzose bist, zweifach in Todesgefahr: einmal, weil Du Franzose, und einmal, weil Du Jude bist. Ich fühle den ganzen Wert einer Gemeinschaft, die keinen Zwiespalt mehr duldet. Wir stammen alle von Frankreich wie aus einer Wurzel, und ich werde Deiner Wahrheit dienen, wie Du der meinen gedient hättest. Für uns Franzosen von draußen handelt es sich in diesem Kriege darum, den Samen der Zukunft aus dem Zustande der Vereisung zu befreien, in den er durch die deutsche Invasion versetzt wurde. Es gilt, Euch da drinnen zu helfen. Es gilt, Euch für die Erde wieder frei zu machen, in der zu wurzeln Ihr das angestammte Recht habt. Ihr seid vierzig Millionen Geiseln. Immer sind es die Keller der Unterdrückung, in denen sich die neuen Wahrheiten vorbereiten: vierzig Millionen Ausgelieferte denken da drinnen über ihre neue Wahrheit nach. Wir unterwerfen uns dieser Wahrheit im voraus.

Denn Ihr werdet es sein, die uns lehren. Es ist nicht an uns, die geistige Flamme jenen zu bringen, die sie schon mit dem Wachs ihrer eigenen Substanz nähren. Ihr lest vielleicht gar nicht unsere Bücher. Ihr hört Euch vielleicht unsere Reden gar nicht an. Vielleicht würdet Ihr unsere Ideen ausspeien. Nicht wir gründen Frankreich. Wir können ihm nur dienen. Was wir auch getan haben mögen, wir werden keinen Anspruch auf Dank haben. Es gibt kein gemeinsames Maß für den freien Kampf und die Vernichtung im Dunkel. Es gibt kein gemeinsames Maß für das Handwerk des Soldaten und den Beruf der Geisel. Ihr seid die Heiligen.


\end{document}

% https://www.projekt-gutenberg.org/saintexu/bekennt/chap001.html
