%!TEX TS-program = Arara
% arara: pdflatex: {shell: yes}
\documentclass[12pt,ngerman]{beamer}

\usepackage[]{blindtext}
\usetheme{Warsaw}
\usepackage[]{graphicx}

\author{Uwe Ziegenhagen}
\title{Meine erste Präsentation}
\subtitle{Untertitel}
\institute{Köln}

\begin{document}

\begin{frame}
\transboxout

\maketitle

\end{frame}

\begin{frame}
\frametitle{Inhalt}

\tableofcontents

\end{frame}

\section{Einleitung}


\begin{frame}
\transdissolve
\frametitle{Warum Beamer cool ist!}

\begin{equation}
a^2+b^2=c^2
\end{equation}

\pause

\begin{equation}
-\frac{p}{2} \pm \sqrt{ \left(\frac{p}{2}\right)^2 -q }
\end{equation}

\end{frame}

\subsection{Aufzählungen}

\begin{frame}
\frametitle{Aufzählungen}

\begin{itemize}
\item<2-> Hallo
\item<1> ich 
\item<1,3> bin
\item<4> eine
\item<2-> längere
\item<-3>Aufzählung
\end{itemize}
\end{frame}

\begin{frame}
\frametitle{Mehrere Spalten}

\begin{columns}
\begin{column}{0.33\textwidth}
\includegraphics[width=\textwidth]{Sampledocument/Images/Katze}
\end{column}
\begin{column}{0.33\textwidth}
\scriptsize\blindtext
\end{column}
\begin{column}{0.33\textwidth}
\scriptsize\blindtext
\end{column}
\end{columns}

\end{frame}

\section{Mehrere Spalten}

\begin{frame}
\frametitle{Mehrere Spalten}

\begin{columns}
\begin{column}{0.66\textwidth}
\includegraphics[width=\textwidth]{Sampledocument/Images/Katze}
\end{column}
\begin{column}{0.33\textwidth}
\begin{itemize}
	\item 
	\item 
	\item 
	\item 
	\item 
	\item 
\end{itemize}


\end{column}
\end{columns}

\end{frame}


\end{document}