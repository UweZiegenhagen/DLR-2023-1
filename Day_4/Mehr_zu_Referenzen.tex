\documentclass[ngerman]{scrartcl}

\usepackage[T1]{fontenc}
\usepackage{babel}
\usepackage{xcolor}
\usepackage{paralist}
\usepackage{todonotes}
% sinnloser Text 
\usepackage{blindtext}
\usepackage[left]{lineno}

\usepackage[right=6cm]{geometry}

\linenumbers

% for prettyref
\usepackage{prettyref} 
\newrefformat{eq}{\textup{(\ref{#1})}}
\newrefformat{cha}{Kapitel \ref{#1}}
\newrefformat{sec}{Abschnitt \ref{#1}}
\newrefformat{tab}{Tabelle \ref{#1} auf Seite \pageref{#1}}
\newrefformat{fig}{Zeichnung \ref{#1} auf der Seite \pageref{#1}}
 
%\usepackage{showlabels} % oder 
\usepackage{showkeys}

\usepackage{hyperref}
\hypersetup{
    bookmarks=true,                     % show bookmarks bar
    unicode=false,                      % non - Latin characters in Acrobat’s bookmarks
    pdftoolbar=true,                        % show Acrobat’s toolbar
    pdfmenubar=true,                        % show Acrobat’s menu
    pdffitwindow=false,                 % window fit to page when opened
    pdfstartview={FitH},                    % fits the width of the page to the window
    pdftitle={My title},                        % title
    pdfauthor={Author},                 % author
    pdfsubject={Subject},                   % subject of the document
    pdfcreator={Creator},                   % creator of the document
    pdfproducer={Producer},             % producer of the document
    pdfkeywords={keyword1, key2, key3},   % list of keywords
    pdfnewwindow=true,                  % links in new window
    colorlinks=true,                        % false: boxed links; true: colored links
    linkcolor=blue,                          % color of internal links
    filecolor=blue,                     % color of file links
    citecolor=blue,                     % color of file links
    urlcolor=blue                        % color of external links
}

\usepackage[noabbrev]{cleveref} % must be loaded after hyperref
%\crefname{figure}{Zeichnung}{Zeichnungen}
% siehe auch https://tex.stackexchange.com/questions/79954/creating-a-new-type-of-cross-reference-and-referencing-them-spelling-the-same-te

\setlength{\marginparwidth}{4cm}

\begin{document}

\listoftodos

\label{aaa}

\section{Erster Abschnitt}\label{sec:erst}
 
\blindtext[1] \todo{Katzenklo säubern}
 
 
\blindtext[1] 
 
\begin{figure}[h]%
\rule{\columnwidth}{5cm}
\caption{Hallo Welt!}%
\label{fig:test1}%\todo{Make a cake \ldots},%
\end{figure}
\clearpage
 
\begin{table}[h]%
\rule{\columnwidth}{5cm}
\caption{Hallo Welt!}%
\label{dra:zeichnung1}%
\end{table}
 
 
\blindtext[1]\marginpar{Hallo, ich bin ein kurzer Text}
 
\section{Zweiter Abschnitt}
 
\blindtext[1]
 
\begin{figure}[h]%
\rule{\columnwidth}{5cm}
\caption{Hallo Welt!}%
\label{fig:test2}%
\end{figure}
 
\blindtext[1]\vspace*{1em}
 
\begin{figure}%
\rule{\columnwidth}{5cm}
\caption{Hallo Welt!}%
\label{fig:hier}%
\end{figure}
 
\begin{compactitem}
	\item Ohne Paket: Siehe Abbildung \ref{fig:test1} auf Seite \pageref{fig:test1} \vspace*{1em}
	\item Cleverref: Siehe \cref{fig:test1}
	\item Cleverref: \namecref{sec:erst} 
	\item Cleverref: \cpageref{fig:test1} 
	\item Cleverref: \crefrange{fig:test1}{fig:test4}
	\item Cleverref: \crefrange*{fig:test1}{fig:test2} % no hyperlink
	\item Cleverref: \cpagerefrange{fig:test1}{fig:test4} 
	\item Cleverref: \cref{dra:zeichnung1}\vspace*{1em} \todo[inline]{DRA definieren}
	\item Prettyref: \prettyref{fig:test1}
	\item Prettyref: \prettyref{sec:erst}
	
\end{compactitem}


 
\clearpage 
 
\begin{figure}%
\rule{\columnwidth}{5cm}
\caption{Hallo Welt!}%
\label{fig:test3}%
\end{figure}
 
\clearpage 
 
\begin{figure}%
\rule{\columnwidth}{5cm}
\caption{Hallo Welt!}%
\label{fig:test4}%
\end{figure}
 
\end{document}

